\documentclass[tikz,border=5pt]{standalone}
\usepackage{tikz}
\usepackage{xeCJK}   
\begin{document}
\begin{tikzpicture}[scale=0.4, thick]

% 坐标轴
\draw[->, gray!70] (-8,0) -- (12,0) node[right] {$x$};
\draw[->, gray!70] (0,-8) -- (0,8) node[above] {$y$};

% ========== 矩形1 ==========
% 中心 (0,0), w1=10, h1=10
\draw[blue, thick] (-5,-5) rectangle (5,5);
\filldraw[blue!20, opacity=0.3] (-5,-5) rectangle (5,5);
\node[blue] at (0,0) {\textbf{A}};
\node[blue, below right] at (-5,-5) {(-5,-5)};
\node[blue, above right] at (5,5) {(5,5)};
\node[blue, below] at (0,-5.5) {宽 $w_1=10$};
\node[blue, rotate=90] at (-5.5,0) {高 $h_1=10$};

% ========== 矩形2 ==========
% 中心 (2,0), w2=6, h2=6
\draw[red, thick] (-1,-3) rectangle (5,3);
\node[red] at (2,0) {\textbf{B}};
\node[red, below right] at (-1,-3) {(-1,-3)};
\node[red, above right] at (5,3) {(5,3)};
\node[red, below] at (2,-3.8) {宽 $w_2=6$};
\node[red, rotate=90] at (-1.8,0) {高 $h_2=6$};


% 辅助线:矩形中心
\draw[dashed, blue] (0,-6) -- (0,6);
\draw[dashed, red] (2,-6) -- (2,6);

% 标注中心点
\fill[blue] (0,0) circle (4pt);
\fill[red] (2,0) circle (4pt);
\node[below right, blue] at (0,0) {$(0,0)$};
\node[below right, red] at (2,0) {$(2,0)$};

\end{tikzpicture}
\end{document}
