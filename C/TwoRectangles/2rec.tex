\documentclass[tikz,border=5pt]{standalone}
\usepackage{tikz}
\usepackage{xeCJK}
\begin{document}
\begin{tikzpicture}[scale=0.4, thick]

% 坐标轴
\draw[->, gray!70] (-8,0) -- (8,0) node[right] {$x$};
\draw[->, gray!70] (0,-8) -- (0,14) node[above] {$y$};

% --- 矩形 A ---
\draw[blue, thick] (-5,-5) rectangle (5,5);
\filldraw[blue!40, opacity=0.2] (-5,-5) rectangle (5,5);
\node[blue] at (0,0) {\textbf{A}};
\draw[dashed, blue] (0,-5) -- (0,5);
\draw[dashed, blue] (-5,0) -- (5,0);
\node[blue, below right] at (-5,-5) {左下角$(-5,-5)$};
\node[blue, above right] at (5,5) {右上角$(5,5)$};
\node[blue] at (6,0) {宽=10};
\node[blue, rotate=90] at (0,6) {高=10};    

% --- 矩形 B ---
\draw[red, thick] (5,5) rectangle (9,9);
\filldraw[red!40, opacity=0.2] (5,5) rectangle (9,9);
\node[red] at (7,7) {\textbf{B}};
\draw[dashed, red] (7,5) -- (7,9);
\draw[dashed, red] (5,7) -- (9,7);
\node[red, below right] at (5,5) {左下角$(5,5)$};
\node[red, above right] at (9,9) {右上角$(9,9)$};


% --- 中心点标记 ---
\filldraw[blue] (0,0) circle (3pt);
\filldraw[red] (7,7) circle (3pt);
\node[blue, left] at (0,0) {$(0,0)$};
\node[red, left] at (7,7) {$(7,7)$};

% --- 说明 ---
\node[gray] at (0,13) {矩形按中心点定义,正好接触于1点};

\end{tikzpicture}
\end{document}
